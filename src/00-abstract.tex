\begin{abstract}
    % the background
    % CARLA is a well-established open-source simulator for high-fidelity emulation of self-driving systems, which has been widely used for the development and validation of machine learning models on autonomous driving.
    %<*tag:abstract>
    \revise{
        Federated learning on vehicles has been widely considered in autonomous driving. The model uploading of federated learning is time-consuming, which motivates the transmission optimization that exploits the mobility of vehicles.
    }%
    In this paper, an optimization framework integrated with the high-fidelity traffic simulator CARLA, namely {\fwName}, is proposed for the scheduling of model uploading in a Vehicle-to-Infrastructure (V2I) network.
    Specifically, a group of {\IAVFullnames} ({\IAVs}) move along their planned routes with random velocities to collect sensor data and train a model via federated learning, where the uplink model transmission of multiple {\IAVs} in one learning iteration is jointly optimized.
    % the platform
    We first develop the {\fwName} simulator from CARLA, to obtain a high-fidelity trajectory dataset of {\IAVs} in the considered traffic scenario, and train the transition probabilities of the {\IAVs}' locations.
    % the problem
    Hence, the uplink transmission time and power of all {\IAVs} in all the time slots can be formulated as a finite-horizon Markov Decision Process (MDP), where the prediction of {\IAVs}' future locations is exploited to minimize a weighted sum of the average model uploading time and the average energy consumption.
    \revise{
        A low-complexity {\fwName} optimizer with a non-trivial performance bound is then proposed, where the optimization consists of global transmission time allocation and local power allocation. The former periodically allocates the transmission time for the remaining time slots according to an average trajectory. Hence, the power allocation of all {\IAVs} can be decoupled to local finite-horizon MDPs in the latter. They can be solved by one-step policy improvement with low computational complexity.
    }%
    The simulation results show that the proposed solution framework can achieve both a significant performance gain over the baselines and a flexible trade-off between performance and complexity.
    %</tag:abstract>
\end{abstract}

\begin{IEEEkeywords}
    Federated edge learning (FEEL),
    CARLA,
    Approximate MDP.
\end{IEEEkeywords}
