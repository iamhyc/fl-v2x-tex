\section{Introduction}
\label{sec:introduction}
%NOTE: (1) federated learning for autonomous driving
Deep learning for autonomous driving \cite{ad-survey} has raised significant communication and computation burdens to wireless edge computing systems. This is because the raw data for model training collected on {\IAVFullnames} ({\IAVs}) is up to terabytes. Fortunately, the vehicular federated learning, e.g., \cite{icra21-invs,feel-wangs}, could suppress the unnecessary cost of data communication by uploading the locally trained models \cite{vfl-survey}. Nevertheless, the model uploading in the federated learning still consumes significant transmission time and energy. In this paper, we would like to investigate the efficient model uploading techniques in a  vehicle-to-infrastructure (V2I) network by exploiting the prediction on {\IAVs}' random trajectories.

%NOTE: (2) related works for FL on vehicles
% deterministic optimization for data uploading or offloading
The communication and computation scheduling for vehicle networks couples tightly with the vehicles' trajectories (the vehicles' locations versus time).
There have been a number of works considering the joint scheduling of communication and edge computing (or edge learning) in vehicle networks with deterministic trajectories
\cite{Globecom18-Wang, Access19-Xu, TITS21-Xiao, JSAC23-Pervej, IOTJ22-Lv, TVT22-Hui, ICC23-Bansal},
where the future locations of vehicles could be planed or known in advance.
For example in \cite{Globecom18-Wang}, the scenario that multiple vehicles move along one unidirectional road at a constant speed and offload the computation tasks to one base station (BS) was considered, where a heuristic scheme was proposed to suppress the delay of task offloading via vehicle-to-vehicle (V2V) communications.
The edge computing scenario with multiple BSs and disjoint coverage was then investigated in \cite{Access19-Xu}. A genetic algorithm was proposed to make the task offloading decision, such that a good trade-off between the total turn-around time and the total resource utility could be achieved.
In \cite{TITS21-Xiao}, the weighted summation of time and energy consumption was minimized in vehicular federated learning, where the velocities of vehicles and the transmission data rate were assumed to be constant. 
Although most of the related works assumed deterministic trajectory, the actual velocities of the vehicles are random, depending on road map, population distribution, working hours, car-holding rate, driving habits and etc. The above scheduling design based on deterministic optimization could not be directly applied, as scheduling with random trajectory might be stochastic optimization problems.

% stochastic optimization for offloading
There have been a few works on the scheduling of data communication or edge computing in vehicle networks with random trajectory
\cite{TVT18-Ni, Access19-Liu, TVT19-Liu, WC16-Salahuddin, TITS16-Wang, FGCS23-Sethi}.
For example in \cite{TVT18-Ni}, a hybrid vehicle-to-vehicle (V2V) and vehicle-to-infrastructure (V2I) network was considered with random inter-vehicle distance, where a heuristic strategy was proposed to cluster the vehicles and maximize the expected communication capacity.
In \cite{Access19-Liu}, continuous task offloading from single vehicle to multiple edge servers was optimized via the Q-learning with the average task turn around time as the minimization objective. However, it is difficult to extend the Q-learning method to multi-vehicle scenario, due to the curse of dimensionality.
In \cite{TVT19-Liu}, the scenario that both fixed and vehicular servers would provide computation service to multiple user equipment was investigated. The Deep Q-Network (DQN) was leveraged to train the offloading and power allocation decisions, such that the average total communication and computation cost of all UEs was minimized. However, the model training might be of significant computational complexity and the performance can hardly be bounded analytically with the method of DQN. Moreover, the models of random traffic in the above works might be too ideal to fit the real applications. Hence, there are still some open issues on the scheduling algorithm design for vehicle networks:
\begin{enumerate}
    \item How to depict the randomness of vehicular trajectory resembling the real-world statistics, instead of assuming it in trivial form like Gaussian distribution?
    \item How to efficiently and reliably handle long-term and large-scale online resource allocation for vehicle networks with random traffic?
\end{enumerate}
% \begin{itemize}
%     \item How to obtain the statistics of the vehicles' random trajectories for one particular road network? In practice, the random trajectories of vehicles may depend on road map, population distribution, working hours, car-holding rate, driving habit and etc. The cost of real measurement is significant.
%     \item Is there any reliable and low-complexity stochastic optimization framework for the transmission scheduling of in-vehicle federated learning? 
% \end{itemize}

To shed some lights on the above issues, we particularly consider the scenario of in-vehicle federated learning, and propose a optimization framework with high-fidelity traffic simulator, namely {\fwName}, for scheduling of the model uploading. In fact, the scheduling of model uploading for multiple vehicles with random trajectories is a finite-horizon Markov Decision Process (MDP), where the complexity of optimal solution grows exponentially with respect to the vehicle number \cite{mdp-cui,mdp-zhou}. The {\fwName} aims to exploit the statistical knowledge on the trajectories of vehicles and provide a low-complexity scheduling solution with a good performance, where the contributions are summarized below.
\begin{itemize}
    \item We develop the {\fwName} simulator to obtain high-fidelity trajectory datasets of {\IAVs} in customizable traffic scenarios, such that the random trajectories of vehicles can be modelled as Markov chains via statistical learning and predicted in online scheduling. 
    \item We propose a novel two-time-scale solution algorithm, namely {\fwName} optimizer, to find a sub-optimal policy with low complexity. Moreover, a non-trivial performance bound is provided. The simulation results show that the {\fwName} optimizer outperforms various baselines, and achieves good balance between computational complexity and performance.
\end{itemize}

The remainder of this paper is organized as follows.
In Section \ref{sec:model}, the traffic model of {\IAVs}, as well as the models of uplink queuing and data transmission, is described.
In Section \ref{sec:formulation}, the communication scheduling problem is formulated as a finite-horizon MDP, where the structure and challenge of optimal solution is discussed.
In Section \ref{sec:framework}, the {\fwName} simulator and the statistical learning for the vehicles' random trajectories are elaborated.
Then, the {\fwName} optimizer is introduced in Section \ref{sec:new_framework}, \ref{sec:kernel-policy} and \ref{sec:local-policy}.
Finally, the numerical simulation is presented in Section \ref{sec:simulation} and the conclusion is drawn in Section \ref{sec:conclusion}.

In this paper, we use the following notations:
non-bold letters (e.g., $a, A$) are used to denote scalar values,
bold lowercase letters (e.g., $\vec{a}$) are used to denote column vectors,
bold uppercase letters (e.g., $\mat{A}$) are used to denote matrices,
and calligraphic letters (e.g., $\mathcal{A}$) are used to denote sets.
$(\mat{A})_{i,j}$ denotes the $(i,j)$-th entry of the matrix $\mat{A}$.
$[a_{j}]_{j}$ with $j\in \mathcal{J}$ denotes the column vector whose entries' indexes take values from sets $\mathcal{J}$ in ascending order; $[a_{i,j}]_{i,j}$ with $i\in \mathcal{I}$ and  $j\in \mathcal{J}$ denotes the matrix.
% $[a_{i,j}]_{i\in \mathcal{I}; j\in \mathcal{J}}$ denotes a matrix whose row and column indexes take values from the sequences $\mathcal{I}$ and $\mathcal{J}$ respectively.
$\|{\vec{a}}\|_2$ denote the L2-norm of vector $\vec{a}$.
$\indicator[\cdot]$ is the indicator function which is equal to $1$ when the inner statement is true and $0$ otherwise.
