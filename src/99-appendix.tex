\appendices

\section{ Proof of Lemma \ref{lemma:local_rate_opt} }
\label{append_1}
The original optimization problem is convex, and we discuss the solution according to the sign of $\Delta{r}_{m,t_i}$. In the case of $\Delta{r}_{m,t_i} \leq 0$, we have $\Delta{r}_{m,\tau} \geq 0$ ($\tau=t_i+1, \dots, T^{(k,*)}$).
Therefore, let $\lambda \in \domR_+$ and $\vecG{\mu} \in \domR^{(T^{(k,*)}-t_i)}_+$ denote the Lagrangian multipliers, the corresponding Lagrangian of the convex optimization problem is given as follows:
\begin{align*}
    L( \Delta{\slide{r}}^{(k,i)}_{m}, \lambda, \vecG{\mu} ) =
    &\sum_{\tau=t_i+1}^{T^{(k,*)}} \gamma^{(k,*)}_{m,\tau} p_{m,\tau}( r^{(k,i)}_{m,\tau} - \Delta{r}_{m,\tau} )
    \nonumber\\
    &+ \lambda \Paren{ \sum_{\tau=t_i+1}^{T^{(k,*)}} \Delta{r}_{m,\tau} - \Delta{r}_{m,t_i} }
    \nonumber\\
    &- \sum_{j=1}^{T^{(k,*)}-t_i} \mu_{j} \Delta{r}_{m,t_i+j}.
\end{align*}

The Karush-Kuhn-Tucker (KKT) \cite{boyd-convex} conditions for the original convex optimization problem are
\begin{enumerate}
    \item \textbf{primal constraints}:
    \begin{align*}
        \sum_{\tau=t_i+1}^{T^{(k,*)}} \Delta{r}_{m,\tau} - \Delta{r}_{m,t_i} &\leq 0,
        \\
        \Delta{\slide{r}}^{(k,i)}_{m} \vecG{\mu}^T &= 0
    \end{align*}
    \item \textbf{dual constraint}:
    $$
    \lambda \geq 0
    $$
    \item \textbf{complementary slackness}:
    $$
    \lambda (\sum_{\tau=t_i+1}^{T^{(k,*)}} \Delta{r}_{m,\tau} - \Delta{r}_{m,t_i}) = 0
    $$
    \item \textbf{stationarity}:
    $$
    \nabla_{\Delta{\slide{r}}^{(k,i)}_{m}} L( \Delta{\slide{r}}^{(k,i)}_{m}, \lambda, \vec{\mu} )
    = \vec{0},
    $$
    i.e.,
    $$
    - \frac{\ln{2}}{T_sB_0} \mathbb{E}[l^{\epsilon}_{m,t_i+j}] \cdot 2^{\frac{r^{(k,i)}_{m,t_i+j} - \Delta{r}_{m,t_i+j}}{T_sB_0 \gamma^{(k,*)}_{m,t_i+j}}} + \lambda - \mu_j = 0,
    \forall j.
    $$
\end{enumerate}

According to $\Delta{\slide{r}}^{(k,i)}_{m} \vecG{\mu}^T = 0$, we have the following statement:
if $\mu_j > 0$, then $\Delta{r}_{m,t_i+j} = 0$ and
$$
\lambda < \frac{\ln{2}}{T_sB_0} \mathbb{E}[l^{\epsilon}_{m,t_i+j}] \cdot 2^{\frac{r^{(k,i)}_{m,t_i+j}}{T_sB_0 \gamma^{(k,*)}_{m,t_i+j}}}, \forall j.
$$
Moreover, we could further conclude that: $\Delta{r}_{m,t_i+j} > 0$ if
$$
\lambda \geq \frac{\ln{2}}{T_sB_0} \mathbb{E}[l^{\epsilon}_{m,t_i+j}] \cdot 2^{\frac{r^{(k,i)}_{m,t_i+j}}{T_sB_0 \gamma^{(k,*)}_{m,t_i+j}}}.
$$
% \begin{itemize}
%     \item if $\mu_j > 0$, then $\Delta{r}_{m,t_i+j} = 0$ and the following inequality holds:
%     $$
%     \lambda < \frac{\ln{2}}{T_sB_0} \mathbb{E}[l^{\epsilon}_{m,t_i+j}] \cdot 2^{\frac{r^{(k,i)}_{m,t_i+j}}{T_sB_0 \gamma^{(k,*)}_{m,t_i+j}}}, \forall j.
%     $$
%     We further conclude that: if $\lambda \geq \frac{\ln{2}}{T_sB_0} \mathbb{E}[l^{\epsilon}_{m,t_i+j}] \cdot 2^{\frac{r^{(k,i)}_{m,t_i+j}}{T_sB_0 \gamma^{(k,*)}_{m,t_i+j}}}$, then $\Delta{r}_{m,t_i+j} > 0$.
%     \item if $\mu_j = 0$, then $\Delta{r}_{m,t_i+j} \geq 0$, and we have
%     $$
%     \Delta{r}_{m,\tau} = - r^{(k,i)}_{m,\tau} + T_sB_0 \gamma^{(k,*)}_{m,\tau} \log_2{\frac{\lambda \cdot T_sB_0}{\ln{2} \mathbb{E}[l^{\epsilon}_{m,\tau}]}}, \forall \tau>t_i.
%     $$
% \end{itemize}
Therefore, $\Delta{r}_{m,t_i+j}$ ($\forall j$) is either positive or zero depending on the scalar value $\lambda$.
Hence, for sufficiently small $\Delta{r}_{m,t_i}$ w.r.t non-trivial optimal solution $\lambda^* > 0$, there is only one non-zero rate allocation whose index is given below.
$$
\hat{\tau}_{m} = \arg\max_{\tau} \mathbb{E}\Bracket{ l_{m,\tau}^{\epsilon}~|~\vec{d}_{m,t_i} }
                          \cdot 2^{\frac{r^{(k,i)}_{m,\tau}}{T_sB_0 \gamma^{(k,*)}_{m,\tau}}}.
$$

Similarly, in the case of $\Delta{r}_{m,t_i} \ge 0$, let $\Delta{\slide{r}}^{(k,i)'}_{m} = - \Delta{\slide{r}}_{m,t_i}$, we have the expression of Lagrangian $L'(\cdot)$ w.r.t $\Delta{\slide{r}}^{(k,i)'}_{m,t_i}$ as follows:
\begin{align*}
    L'( \Delta{\slide{r}}^{(k,i)'}_{m}, \lambda, \vecG{\mu} ) =
    &\sum_{\tau=t_i+1}^{T^{(k,*)}} \gamma^{(k,*)}_{m,\tau} p_{m,\tau}( r^{(k,i)}_{m,\tau} + \Delta{r}'_{m,\tau} )
    \nonumber\\
    &- \lambda \Paren{ \sum_{\tau=t_i+1}^{T^{(k,*)}} \Delta{r}_{m,\tau} - \Delta{r}_{m,t_i} }
    \nonumber\\
    &- \sum_{j=1}^{T^{(k,*)}-t_i} \mu_{j} \Delta{r}_{m,t_i+j}.
\end{align*}
Therefore, the only difference compared to the analysis in previous case exists in stationarity constraint, where
$$
\frac{\ln{2}}{T_sB_0} \mathbb{E}[l^{\epsilon}_{m,t_i+j}] \cdot 2^{\frac{r^{(k,i)}_{m,t_i+j} - \Delta{r}'_{m,t_i+j}}{T_sB_0 \gamma^{(k,*)}_{m,t_i+j}}} + \lambda - \mu_j = 0,
    \forall j.
$$
Hence, for sufficiently small $\Delta{r}_{m,t_i}$, the index of the non-zeros rate allocation is the {\it smallest one} which satisfies the corresponding constraint, i.e.,
$$
\breve{\tau}_{m} = \arg\min_{\tau} \mathbb{E}\Bracket{ l_{m,\tau}^{\epsilon}~|~\vec{d}_{m,t_i} }
                          \cdot 2^{\frac{r^{(k,i)}_{m,\tau}}{T_sB_0 \gamma^{(k,*)}_{m,\tau}}}.
$$

\section{ Proof of Lemma \ref{lemma:local_approx_solution} }
\label{append_2}
Given the selected time slot $\tau$ in Lemma \ref{lemma:local_rate_opt}, the corresponding optimal rate allocation is given as follows:
\begin{align*}
    \min_{ \Delta{r}_{m,t_i} } &\gamma^{(k,*)}_{m,t_i} p_{m,t_i}( r^{(k,i)}_{m,t_i} + \Delta{r}_{m,t_i} )
    \nonumber\\
    &+ \gamma^{(k,*)}_{m,\tau} p_{m,\tau}( r^{(k,i)}_{m,\tau} - \Delta{r}_{m,t_i} )
    % \nonumber\\
    % \text{s.t.~}& \Delta{r}_{m,t_i} < r^{(k,i)}_{m,\tau}
    % \nonumber\\
    % & -\Delta{r}_{m,t_i} < r^{(k,i)}_{m,t_i}
\end{align*}
The solution to the above optimization problem is easy to obtain by taking its derivative w.r.t $\Delta{r}_{m,t_i}$ and setting it to zero, i.e.,
$$
\frac{\ln{2}}{T_sB_0} l^{\epsilon}_{m,t_i} 2^{\frac{r^{(k,i)}_{m,t_i} + \Delta{r}_{m,t_i}}{T_sB_0 \gamma^{(k,*)}_{m,t_i}}}
+ \frac{\ln{2}}{T_sB_0} \mathbb{E}[l^{\epsilon}_{m,\tau}] 2^{\frac{r^{(k,i)}_{m,\tau} - \Delta{r}_{m,t_i}}{T_sB_0 \gamma^{(k,*)}_{m,\tau}}} = 0,
$$
and the solution is as follows:
\begin{align*}
    \Delta{r}_{m,t_i} =
    &\frac{
        T_sB_0 \gamma^{(k,*)}_{m,t_i} \gamma^{(k,*)}_{m,\tau} \paren{ \log_2{ \mathbb{E}[l^{\epsilon}_{m,\tau}] } - \log_2{l^{\epsilon}_{m,t_i}} } 
    }{
        \gamma^{(k,*)}_{m,t_i} + \gamma^{(k,*)}_{m,\tau}
    }
    \nonumber\\
    &+ \frac{
        r^{(k,*)}_{m,\tau}\gamma^{(k,*)}_{m,t_i} - r^{(k,*)}_{m,t_i}\gamma^{(k,*)}_{m,\tau} 
    }{
        \gamma^{(k,*)}_{m,t_i} + \gamma^{(k,*)}_{m,\tau}
    }.
\end{align*}

\section{ Proof of Theorem \ref{theorem:performance_analysis} }
\label{append_3}
In the following, we show that each global or local optimization of transmit time or power according to the current system state will suppress the system cost.
In the $t_i$-th time slot where $t_i=kN+i$ ($\forall k,i$),
let 
\begin{align*}
    \bar{\Policy} = \Brace{ \gamma^{(k,*)}_{m,\tau}, r^{(k,i-1)}_{m,\tau} | \forall m\in\carSet, \tau=t_i,\dots,T^{(k,*)} }
\end{align*}
denotes the policy on the right-hand-side of equation \eqref{eqn:performance_bound}, which is a state-invariant policy.
Let $\mat{L} \define \paren{ \vec{d}_{t_i}, \dots, \vec{d}_{\T} }$
% \begin{align*}
%     \mat{L} \define \Paren{ \vec{d}_{t_i}, \dots, \vec{d}_{\T} }
% \end{align*}
denotes the trajectory of random system states of {\IAVs} from the $t_i$-th time slot to the maximum scheduling period $\T$ where $\vec{d}_{\tau} = [ \vec{d}_{m,\tau} ]_m$ ($\forall \tau$),
and $\Baseline_{L}$ denotes the fixed proposed policy along the trajectory $\mat{L}$.
Therefore, for the fixed trajectory $\mat{\bar{l}} = ( \bar{\vec{d}}_{t_i}, \dots, \bar{\vec{d}}_{\T} )$, the proposed policy $\Baseline_{\bar{l}}$ updates the resource allocation only once at the $t_i$-th slot and we have:
\begin{align*}
    V^{\Baseline_{\bar{l}}}_{t_i}( \Stat_{t_i} ) &= g_{t_i}\Paren{ \Stat_{t_i}, \Baseline_{\bar{l}}(\Stat_{t_i}) } + \mathbb{E}\bracket{ V^{ \Baseline_{\bar{l}} }( \Stat_{t_i+1} ) }
    \nonumber\\
    &\leq g_{t_i}\Paren{ \Stat_{t_i}, \bar{\Policy}(\Stat_{t_i}) } + \mathbb{E}\bracket{ V^{ \bar{\Policy} }( \Stat_{t_i+1} ) }
    \nonumber\\
    &= T^{(k,*)} + \omega \sum_{\tau=t_i, m} \gamma^{(k,*)}_{m,\tau} \mathbb{E}\bracket{ p( \gamma^{(k,*)}_{m,\tau}, r^{(k,i-1)}_{m,\tau} ) }
\end{align*}
Moreover, let
$$
\mat{L}^{(n)} = ( \bar{\vec{d}}_{t_i}, \dots, \bar{\vec{d}}_{t_i+n-1}, \vec{d}_{t_i+n}, \dots,  \vec{d}_{\T})
$$
denote a series of trajectories where the first $n$ elements are fixed and the rest are random ($n=1,\dots,\T-t_i+1$).
Specifically, $\mat{L}^{(0)} = \mat{L}$, and $\mat{L}^{(\T-t_i+1)} = \bar{l}$ is the fixed expected trajectory.
According to the policy improvement in the solution to Problem \textbf{P2$(k)$} at the start of super slot and Problem \textbf{P3$(k,m)$} for local rate allocation, the policy update between any two adjacent time slot resembles the \emph{policy improvement} step, and we have
$V^{\Baseline_{\mat{L}^{(n)}}}_{t_i}( \Stat_{t_i} ) \leq V^{\Baseline_{\mat{L}^{(n+1)}}}_{t_i}( \Stat_{t_i} )$.
Therefore, we have the following inequality holds:
\begin{align*}
    V^{\Baseline_{\mat{L}}}_{t_i}( \Stat_{t_i} ) &\leq V^{\Baseline_{\mat{L}^{(1)}}}_{t_i}( \Stat_{t_i} )
    \nonumber\\
    &\leq \dots
    \nonumber\\
    &\leq
    V^{\Baseline_{\mat{L}^{(n)}}}_{t_i}( \Stat_{t_i} ) \leq
    V^{\Baseline_{\mat{L}^{(n+1)}}}_{t_i}( \Stat_{t_i} )
    \nonumber\\
    &\leq \dots
    \nonumber\\
    &\leq V^{\Baseline_{\mat{L}^{(\T-t_i)}}}_{t_i}( \Stat_{t_i} ) \leq V^{\Baseline_{\bar{l}}}_{t_i}( \Stat_{t_i} ).
\end{align*}
As a result, the inequality in equation \eqref{eqn:performance_bound} holds, i.e.,
\begin{align*}
    V^{\Baseline}_{t_i}( \Stat_{t_i} ) &\leq V^{\Baseline_{\bar{l}}}_{t_i}( \Stat_{t_i} )
    \nonumber\\
    &\leq T^{(k,*)} + \omega \sum_{\tau=t_i, m} \gamma^{(k,*)}_{m,\tau} \mathbb{E}\bracket{ p( \gamma^{(k,*)}_{m,\tau}, r^{(k,i-1)}_{m,\tau} ) }.
\end{align*}
